% !TEX root = ../main.tex
%---------------------------------------------------------------------------------------------------
%---------------------------------------------------------------------------------------------------
\section{Decision-making with models}
%---------------------------------------------------------------------------------------------------
%---------------------------------------------------------------------------------------------------
\begin{frame}\frametitle{Decision-making with models}\vspace{0.1cm}\centering
\resizebox {0.45\textwidth} {!} {
  % Decision-making with models across domains


\begin{tikzpicture}
   \fill[even odd rule, mediumblue!20, draw=mediumblue!30] circle (4);
   \fill[even odd rule, mediumblue] circle (1.7);

   \node at (0,0) [
      color = white,
      align = center
   ]{
      Decision-making\\
      with models
   };
   \arcarrow{ 85}{  3}{ Statistics  }
   \arcarrow{270}{357}{ Mathematics }
   \arcarrow{182}{269}{ Operations  }
   \arcarrow{176}{ 96}{ Economics   }
   \arctext{180}{0}{ Uncertainty    }
\end{tikzpicture}

  }
\end{frame}
%---------------------------------------------------------------------------------------------------
%---------------------------------------------------------------------------------------------------
\begin{frame}\frametitle{Eckstein--Keane--Wolpin models}

\begin{multicols}{2}

	\heading{Understanding individual decisions}\vspace{0.3cm}
	\begin{itemize}\setlength\itemsep{1em}
		\item Human capital investment
		\item Consumption--savings decision
	\end{itemize}

	\heading{Predicting effects of policies}\vspace{0.3cm}
	\begin{itemize}\setlength\itemsep{1em}
		\item Educational policy
		\item Welfare programs
	\end{itemize}

\end{multicols}

\pause
\heading{Core economic parameters}\vspace{0.3cm}
\begin{itemize}\setlength\itemsep{1em}
 \item Time preferences
 \item Returns to schooling
 \item Returns to work-experience
\end{itemize}

\end{frame}
%---------------------------------------------------------------------------------------------------
%---------------------------------------------------------------------------------------------------
\begin{frame}{As-if analysis}\vspace{0.3cm}

In economics, however, we use the point estimates as a plug-in for the true parameter and the model is analyzed as-if the true parameters are known \citep{Manski.2021}.\vspace{0.3cm}
\pause
  	\heading{Consequences}
    \vspace{0.3cm}
  	\begin{itemize}\setlength\itemsep{1em}
  	\item Fragile findings as facts
  	\item Dueling certitudes stifle constructive debate
  	\item Knowledge gaps are not identified
    \item Policy advice not framed as a decision problem under uncertainty
  	\end{itemize}

\end{frame}
%---------------------------------------------------------------------------------------------------
%---------------------------------------------------------------------------------------------------
\begin{frame}{Embracing statistical decision theory}\vspace{0.4cm}

  \begin{columns}
  	\begin{column}{0.55\textwidth}

  \begin{itemize}\setlength\itemsep{1em}
  \item Promote a well-reasoned and transparent decision-making process
  \item Clarify trade-off between actions
  \item Facilitate communication of uncertainty
\end{itemize}\vspace{0.3cm}
\end{column}

\begin{column}{0.4\textwidth}
\vspace{-0.4cm}
\begin{figure}
\includegraphics[scale=0.18]{../material/abraham-wald}
\caption*{\textit{Abraham Wald}}
\end{figure}
\end{column}
\end{columns}
\vspace{-2.4cm}

\only<2>{\large $\Rightarrow$\alert<2>{ Conceptually simple, computationally challenging}}

\end{frame}
%---------------------------------------------------------------------------------------------------
%---------------------------------------------------------------------------------------------------
\begin{frame}\frametitle{Basic setup}\vspace{0.3cm}


  \heading{Consequence function}\vspace{-0.3cm}
  \begin{align*}
     c = \rho(a, \theta)
  \end{align*}

  \pause
  \heading{Notation}
  \begin{columns}
  \begin{column}{0.5\textwidth}
    \begin{align*}\begin{array}{ll}
    \theta \in \Theta & \text{parameters}\\[0.5em]
    c \in C & \text{consequence} \\[0.5em]
    a \in A & \text{action}  \\[0.5em]
    \end{array}\end{align*}
  \end{column}
  \begin{column}{0.5\textwidth}
    \begin{align*}\begin{array}{ll}
      \psi \in\Psi & \text{sample data} \\[0.5em]
      \delta \in \Gamma     & \text{decision function}  \\[0.5em]
    P_\theta & \text{sampling distribution} \\[0.5em]
    \end{array}\end{align*}
  \end{column}
  \end{columns}

\end{frame}
%---------------------------------------------------------------------------------------------------
%---------------------------------------------------------------------------------------------------
\begin{frame}{Statistical decision functions}\vspace{0.3cm}

\begin{figure}[h!]\centering
\scalebox{0.75}{ \centering
 \begin{tikzpicture}[ele/.style={fill=black,circle,minimum width=5pt,inner sep=0.5pt},every fit/.style={ellipse,draw,inner sep=18pt}]

\node (data) at (-1,6.2) {Data};
\node (decisions) at (4,6.2) {Decisions};

\node[] (a1) at (-1,4) {};
\node[] (a2) at (-1,1) {};


\node[] (b1) at (4,4) {};
\node[] (b2) at (4,1) {};

\node[fill,gray!20,fit= (a1) (a2),minimum width=2cm] {} ;
\node[fill,gray!20,fit= (b1) (b2),minimum width=2cm] {} ;

\node[draw, gray,fit= (a1) (a2),minimum width=2cm] {} ;
\node[draw, gray, fit= (b1) (b2),minimum width=2cm] {} ;



\node[ele] (d0) at (-1,2.5) {};


\node[ele, label=right:{\scriptsize $a_1$}] (d1) at (3.5,4) {};
\node[ele, label=right:{\scriptsize $a_2$}] (d2) at (3.5,2.5) {};
\node[ele, label=right:{\scriptsize $a_3$}] (d3) at (3.5,1) {};



  \draw[->,thick,shorten <=2pt,shorten >=2pt] (d0) -- node[label={[xshift=-0.2cm, yshift=-0.35cm]\rotatebox{15}{$\delta_1$}}] {}(d1);
  \draw[->,thick,shorten <=2pt,shorten >=2pt] (d0) -- node[label={[xshift=0cm, yshift=-0.3cm]{$\delta_2$}}] {}(d2);


  \draw[->,thick,shorten <=2pt,shorten >=2pt] (d0) -- node[label={[xshift=0cm, yshift=-0.4cm]\rotatebox{-15}{$\delta_3$}}] {}(d3);


 \draw[->,thick,shorten <=2pt,shorten >=2pt] (data) -- node[label={[xshift=0cm, yshift=-0.2cm]\footnotesize{SDFs}}] {}(decisions);

 \end{tikzpicture}
}
\end{figure}
\end{frame}
%---------------------------------------------------------------------------------------------------
%---------------------------------------------------------------------------------------------------
\begin{frame}{Comparing actions}\vspace{0.4cm}
\citet{Wald.1950} suggests measuring the performance of $\delta$ at all possible parametrizations $\theta$ by computing the expected utility with respect to its induced sampling distribution $P_\theta$:

\begin{align*}
    \E_{\theta}\left[u\left(\rho(\delta(\psi), \theta)\right)\right] = \int_\Psi u\left(\rho(\delta(\psi), \theta)\right) d P_{\theta}(\psi).
\end{align*}

\end{frame}
%---------------------------------------------------------------------------------------------------
%---------------------------------------------------------------------------------------------------
\begin{frame}{Decision-theoretic criteria}\vspace{0.6cm}


  \begin{itemize}
    \item \makebox[0pt][l]{Maximin}\phantom{spacespacespacespace} $\quad \delta^*= \argmax_{\delta \in \Gamma } \min_{\theta\in \Theta} \E_{\theta}\left[u(\rho\left(\delta(\psi), \theta\right))\right]$
    \item []
    \item \makebox[0pt][l]{Minimax regret}\phantom{spacespacespacespace} $\quad \delta^* =  \argmin_{\delta \in \Gamma } \max_{\theta\in \Theta}   \underbrace{\left[\max_{a \in A}  u(\rho\left(a, \theta\right))  - \E_{\theta}\left[u(\rho\left(\delta(\psi), \theta\right)) \right]\right]}_{\text{regret}}$
    \item []
    \item \makebox[0pt][l]{Subjective Bayes}\phantom{spspacespacespacespace} $\delta^* = \argmax_{\delta \in \Gamma }  \int_{\theta} \E_{\theta}\left[u(\rho\left(\delta(\psi), \theta\right))\right]df_{\theta}$
  \end{itemize}

\end{frame}
%---------------------------------------------------------------------------------------------------
%---------------------------------------------------------------------------------------------------
